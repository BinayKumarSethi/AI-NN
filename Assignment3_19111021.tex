\documentclass[10pt,a4paper,twoside]{article}
\usepackage[dutch]{babel}
\usepackage{amssymb}
\usepackage{amsmath}
\usepackage{float,flafter}	
\usepackage{hyperref}
\usepackage{inputenc}
\setlength\paperwidth{20.999cm}\setlength\paperheight{31.699cm}\setlength\voffset{-1in}\setlength\hoffset{-1in}\setlength\topmargin{.1cm}\setlength\headheight{12pt}\setlength\headsep{0cm}\setlength\footskip{1.131cm}\setlength\textheight{29cm}\setlength\oddsidemargin{2.499cm}\setlength\textwidth{15.999cm}

\begin{document}
\begin{center}
\hrule

\vspace{.4cm}
{\bf {\Large ASSIGNMENT-3 }}\\
\vspace{.3cm}
{\bf {\huge MORAVEC'S PARADOX}}
\vspace{.3cm}
\end{center}
{\bf Name:}  BINAY KUMAR SETHI\\
{\bf Roll no:}  19111021 \\
{\bf Branch: }  Biomedical Engineering \hspace{\fill}  22 July, 2021 \\
\hrule

\vspace{.3cm}
\section{Moravec's Paradox} 
It is the observation by artificial intelligence and robotics researchers that, contrary to traditional assumptions, reasoning requires very little computation, but sensorimotor skills require enormous computational resources. He said that it is easy to make a computer perform intelligence test or play smart games flawlessly but still they cant come up with skills of perception and mobility of a one year old.  \\


\section{The Biological Basis Of Human Skills}
A billion years of experience about the nature of the world and how to survive it is encoded in the huge, highly evolved sensory and motor regions of the human brain. The difficulty of reverse-engineering any human skill should be roughly proportionate to the length of time the skill has evolved in animals. The deliberate process we call reasoning is, in my opinion, the thinnest layer of human mind, and it is only effective because it is underpinned by this much older and far more powerful sensory knowledge, which is usually unconscious.The difficulty of reverse-engineering any human skill should be roughly proportionate to the length of time the skill has evolved in animals. As a result, we should expect talents that appear to be simple to reverse-engineer to be difficult to reverse-engineer, whereas skills that involve effort may not be difficult to reverse-engineer at all.

\section{Historical Influence On Artificial Intelligence}
In the early days of artificial intelligence research, leading researchers often predicted that they would be able to create thinking machines in just a few decades.\\ 
The fact that they had solved problems like logic and algebra was irrelevant, because these problems are extremely easy for machines to solve. \\
Rodney Brooks explains that, according to early AI research, intelligence was "best characterized as the things that highly educated male scientists found challenging", such as chess, symbolic integration, proving mathematical theorems and solving complicated word algebra problems.This would lead Brooks to pursue a new direction in artificial intelligence and robotics research. He decided to build intelligent machines that had "No cognition. Just sensing and action. That is all I would build and completely leave out what traditionally was thought of as the intelligence of artificial intelligence."[5] This new direction, which he called "Nouvelle AI" was highly influential on robotics research and AI.










\end{document}
